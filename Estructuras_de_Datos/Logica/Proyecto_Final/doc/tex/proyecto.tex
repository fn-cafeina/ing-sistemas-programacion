\documentclass[12pt, letterpaper]{article}

% --- PAQUETES ESENCIALES ---
\usepackage[utf8]{inputenc}
\usepackage[T1]{fontenc}
\usepackage[spanish]{babel}
\usepackage{geometry}
\usepackage{graphicx}
\usepackage{hyperref}
\usepackage{parskip}
\usepackage{csquotes}

% --- PAQUETES PARA CÓDIGO ---
\usepackage{listings}
\usepackage{xcolor}

\definecolor{codegreen}{rgb}{0,0.6,0}
\definecolor{codegray}{rgb}{0.5,0.5,0.5}
\definecolor{codepurple}{rgb}{0.58,0,0.82}
\definecolor{backcolour}{rgb}{0.95,0.95,0.95}

\lstdefinestyle{cstyle}{
    backgroundcolor=\color{backcolour},
    commentstyle=\color{codegreen},
    keywordstyle=\color{magenta},
    numberstyle=\tiny\color{codegray},
    stringstyle=\color{codepurple},
    basicstyle=\ttfamily\footnotesize,
    breakatwhitespace=false,
    breaklines=true,
    captionpos=b,
    keepspaces=true,
    numbers=left,
    numbersep=5pt,
    showspaces=false,
    showstringspaces=false,
    showtabs=false,
    tabsize=2,
    language=C
}
\lstset{style=cstyle}

% --- PAQUETES PARA DIAGRAMAS (TikZ) ---
\usepackage{tikz}
\usetikzlibrary{shapes.geometric, arrows, positioning, fit}

% --- PAQUETES PARA BIBLIOGRAFÍA (APA 7) ---
\usepackage[
    backend=biber,
    style=apa,
    sorting=nyt
]{biblatex}
\addbibresource{bibliografia.bib}

% --- CONFIGURACIÓN DE PÁGINA ---
\geometry{
    letterpaper,
    left=2.5cm,
    right=2.5cm,
    top=2.5cm,
    bottom=2.5cm
}
\hypersetup{
    colorlinks=true,
    linkcolor=black,
    urlcolor=blue,
    pdftitle={Proyecto 2: Sistema de Gestión de Solicitudes de Mantenimiento},
    pdfauthor={Autores del Proyecto}
}

\title{Proyecto 2: Sistema de Gestión de Solicitudes de Mantenimiento}
\author{Jasmir Moisés Medina Pérez}
\date{7 de noviembre de 2025}

% --- INICIO DEL DOCUMENTO ---
\begin{document}

% --- RESUMEN ---
\section*{Resumen del Proyecto}
Este documento detalla el diseño, desarrollo e implementación del "Sistema de Gestión de Solicitudes de Mantenimiento". El sistema, desarrollado en C, permite registrar, asignar y dar seguimiento a solicitudes de mantenimiento, utilizando archivos binarios para la persistencia de datos.

\tableofcontents
\newpage

% --- OBJETIVOS ---
\section{Objetivos}
\subsection{Objetivos de Aprendizaje}
[cite_start]Los objetivos de aprendizaje definidos para este proyecto son [cite: 52-55]:
\begin{itemize}
    \item Modelar procesos administrativos mediante estructuras en C.
    \item Implementar lógica condicional y ciclos para la gestión de estados.
    \item Utilizar archivos para guardar y recuperar información de manera persistente.
    \item Diseñar una interfaz de usuario funcional en consola.
\end{itemize}

\subsection{Objetivos del Proyecto}
[cite_start]Los objetivos específicos del sistema son [cite: 56-59]:
\begin{itemize}
    \item Registrar solicitudes de mantenimiento (prioridad y ubicación).
    \item Registrar técnicos disponibles.
    \item Asignar solicitudes pendientes a técnicos.
    \item Actualizar el estado de las solicitudes (Pendiente, En Proceso, Finalizada).
    \item Generar reportes por técnico, área o estado.
\end{itemize}


% --- DIAGRAMAS ---
\section{Diagramas del Sistema}

\subsection{Diagrama de Flujo (Menú Principal)}
Diagrama de flujo del bucle principal del programa.

\begin{figure}[h]
\centering
\begin{tikzpicture}[
    node distance=2cm, auto,
    block/.style={rectangle, draw, fill=blue!10, text width=10em, text centered, rounded corners, minimum height=3em},
    loop/.style={rectangle, draw, fill=orange!10, text width=9em, text centered, rounded corners, minimum height=3em},
    decision/.style={diamond, draw, fill=green!10, text width=6em, text centered, minimum height=3em},
    io/.style={trapezium, trapezium left angle=70, trapezium right angle=110, draw, fill=gray!10, text width=8em, text centered, minimum height=3em},
    line/.style={draw, -latex'}
]
    % Nodos
    \node [block] (init) {Inicio: Cargar Datos()};
    \node [loop, below=of init] (menu) {Mostrar Menú Principal};
    \node [io, below=of menu] (input) {Leer Opción (0-5)};
    \node [decision, below=of input] (decide) {Opción == 0?};
    \node [block, right=of decide, xshift=4cm] (op1) {Registrar Solicitud};
    \node [block, right=of op1, xshift=1cm] (op2) {Registrar Técnico};
    \node [block, below=of op1] (op3) {Asignar Tarea};
    \node [block, below=of op2] (op4) {Actualizar Estado};
    \node [block, below=of op3] (op5) {Mostrar Reportes};
    \node [fit=(op1) (op2) (op3) (op4) (op5), label=above:Gestión de Opciones] (switch) {};
    \node [block, below=of decide] (exit) {Guardar Datos() y Salir};

    % Conexiones
    \path [line] (init) -- (menu);
    \path [line] (menu) -- (input);
    \path [line] (input) -- (decide);
    \path [line] (decide) -- node [right, xshift=0.2cm] {No} (op1);
    \path [line] (decide) -- node [left, xshift=-0.2cm] {Sí} (exit);
    \path [line] (op1) |- (menu); \path [line] (op2) |- (menu);
    \path [line] (op3) |- (menu); \path [line] (op4) |- (menu);
    \path [line] (op5) |- (menu);
\end{tikzpicture}
\caption{Diagrama de flujo del bucle principal.}
\label{fig:flowchart}
\end{figure}

\subsection{Diagrama de Estructuras}
Definición de las estructuras base del sistema.
\lstinputlisting[language=C, caption={Definición de estructuras en include/tipos.h}, firstline=14, lastline=34]{../../include/tipos.h}

\subsection{Diagrama de Estructura del Sistema}
Organización de los archivos del proyecto.
\begin{verbatim}
. (Raiz del Proyecto)
|-- doc/
|   |-- requisitos/
|   |   `-- Proyecto Final - AED - GP5.pdf
|   |-- tex/
|   |   |-- proyecto.tex       (Este documento)
|   |   `-- bibliografia.bib
|   `-- proyecto.pdf       (Salida compilada)
|
|-- include/           (Archivos de cabecera .h)
|-- src/               (Archivos fuente .c)
|-- data/              (Datos binarios .dat)
`-- Makefile           (Compilador C)
\end{verbatim}


\newpage
% --- MANUALES ---
\section{Manuales}

\subsection{Manual Técnico}
Descripción de los componentes internos del sistema.

\subsubsection{Módulos y Funciones Clave}
\begin{description}
    \item[\texttt{include/tipos.h}] Define \texttt{tecnico\_t} y \texttt{solicitud\_t}.
    \item[\texttt{include/utils.h}] Provee \texttt{limpiar\_pantalla()}, \texttt{pausar()}, \texttt{leer\_opcion()}.
    \item[\texttt{include/persistencia.h}] Provee \texttt{cargar\_datos()} y \texttt{guardar\_datos()}.
    \item[\texttt{include/logica.h}] Provee \texttt{registrar\_solicitud()}, \texttt{asignar\_tarea()}, etc.
    \item[\texttt{src/main.c}] Contiene el \texttt{main} y el bucle principal del menú.
\end{description}

\subsubsection{Archivos de Datos}
El sistema usa dos archivos binarios (en la carpeta \texttt{data/}):
\begin{itemize}
    \item \texttt{data/tecnicos.dat}: Almacena los técnicos.
    \item \texttt{data/solicitudes.dat}: Almacena las solicitudes.
\end{itemize}
Ambos archivos guardan primero un \texttt{int} (el contador) y un \texttt{int} (siguiente ID), seguido por el bloque de $N$ estructuras.

\subsection{Manual de Usuario}
El sistema se inicia ejecutando \texttt{./bin/gestor\_mantenimiento} desde la terminal raíz.
\begin{description}
    \item[1. Registrar Nueva Solicitud] Pide ubicación, descripción y prioridad.
    \item[2. Registrar Nuevo Técnico] Pide nombre y especialidad.
    \item[3. Asignar Tarea a Técnico] Muestra listas para asignar una solicitud pendiente a un técnico.
    \item[4. Actualizar Estado de Solicitud] Permite cambiar el estado (Pendiente, En Proceso, Finalizada).
    \item[5. Ver Reportes] Submenú para filtrar por Técnico, Área o Estado.
    \item[0. Guardar y Salir] Guarda todos los cambios en los archivos \texttt{.dat}.
\end{description}

\newpage
% --- DESARROLLO Y PRUEBAS ---
\section{Desarrollo y Pruebas}

\subsection{Fragmentos Clave del Código}

\subsubsection{Gestión de Memoria Dinámica}
Uso de \texttt{realloc} para expandir el array global al registrar un técnico.
\lstinputlisting[language=C, caption={Función registrar\_tecnico en src/logica.c}, firstline=99, lastline=121]{../../src/logica.c}

\subsubsection{Validación y Creación de Directorio}
Verificación del directorio \texttt{data/} antes de leer o escribir.
\lstinputlisting[language=C, caption={Función de persistencia en src/persistencia.c}, firstline=15, lastline=29]{../../src/persistencia.c}

\subsection{Pruebas Realizadas}
Resumen de pruebas de caja negra funcionales.

\begin{table}[h]
\centering
\begin{tabular}{|p{4cm}|p{5cm}|p{5cm}|}
\hline
\textbf{Caso de Prueba} & \textbf{Acción Realizada} & \textbf{Resultado Esperado} \\
\hline
Registro (Solicitud/Técnico) & Opción 1 y 2. Ingresar datos. & Ítems creados con ID 1. \\
\hline
Asignación de Tarea & Opción 3. Asignar Sol. 1 a Tec. 1. & Sol. 1 -> "En Proceso", asignada a Tec. 1. \\
\hline
Actualización de Estado & Opción 4. Cambiar Sol. 1 a "Finalizada". & Sol. 1 -> "Finalizada", técnico desasociado. \\
\hline
Persistencia de Datos & Opción 0 (Salir). Volver a ejecutar. & Carga Sol. 1 y Tec. 1. Siguiente ID es 2. \\
\hline
Validación de Entrada & Opción de menú. Ingresar "9" o "abc". & Mensaje de error, vuelve a pedir la opción. \\
\hline
\end{tabular}
\caption{Resumen de casos de prueba funcionales.}
\label{tab:pruebas}
\end{table}


% --- CONCLUSIONES Y MEJORAS ---
\section{Conclusiones y Mejoras Futuras}

\subsection{Conclusiones}
El proyecto se completó, cumpliendo todos los requisitos funcionales. La arquitectura modular facilitó el desarrollo. La gestión de estado y la persistencia en archivos binarios demostraron ser soluciones eficaces en C estándar.

\subsection{Mejoras Futuras}
\begin{itemize}
    \item \textbf{Interfaz Gráfica (GUI):} Reemplazar la consola por una interfaz gráfica.
    \item \textbf{Gestión de Errores:} Mejorar la robustez del sistema ante fallos de I/O.
    \item \textbf{Autenticación de Usuarios:} Añadir un sistema simple de login para diferenciar roles.
\end{itemize}


% --- BIBLIOGRAFÍA ---
\newpage
\printbibliography[title={Bibliografía}]

\end{document}